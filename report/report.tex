% !TEX TS-program = pdflatex
% !TEX encoding = UTF-8 Unicode

% This is a simple template for a LaTeX document using the "article" class.
% See "book", "report", "letter" for other types of document.

\documentclass[11pt]{article} % use larger type; default would be 10pt

\usepackage[utf8]{inputenc} % set input encoding (not needed with XeLaTeX)

%%% Examples of Article customizations
% These packages are optional, depending whether you want the features they provide.
% See the LaTeX Companion or other references for full information.

%%% PAGE DIMENSIONS
\usepackage{geometry} % to change the page dimensions
\geometry{a4paper} % or letterpaper (US) or a5paper or....
% \geometry{margin=2in} % for example, change the margins to 2 inches all round
% \geometry{landscape} % set up the page for landscape
%   read geometry.pdf for detailed page layout information

\usepackage{graphicx} % support the \includegraphics command and options

% \usepackage[parfill]{parskip} % Activate to begin paragraphs with an empty line rather than an indent

%%% PACKAGES
\usepackage{booktabs} % for much better looking tables
\usepackage{array} % for better arrays (eg matrices) in maths
\usepackage{paralist} % very flexible & customisable lists (eg. enumerate/itemize, etc.)
\usepackage{verbatim} % adds environment for commenting out blocks of text & for better verbatim
\usepackage{subfig} % make it possible to include more than one captioned figure/table in a single float
% These packages are all incorporated in the memoir class to one degree or another...

%%% HEADERS & FOOTERS
\usepackage{fancyhdr} % This should be set AFTER setting up the page geometry
\pagestyle{fancy} % options: empty , plain , fancy
\renewcommand{\headrulewidth}{0pt} % customise the layout...
\lhead{}\chead{}\rhead{}
\lfoot{}\cfoot{\thepage}\rfoot{}

%%% SECTION TITLE APPEARANCE
\usepackage{sectsty}
\allsectionsfont{\sffamily\mdseries\upshape} % (See the fntguide.pdf for font help)
% (This matches ConTeXt defaults)

%%% ToC (table of contents) APPEARANCE
\usepackage[nottoc,notlof,notlot]{tocbibind} % Put the bibliography in the ToC
\usepackage[titles,subfigure]{tocloft} % Alter the style of the Table of Contents
\renewcommand{\cftsecfont}{\rmfamily\mdseries\upshape}
\renewcommand{\cftsecpagefont}{\rmfamily\mdseries\upshape} % No bold!

 \setlength{\parindent}{0pt}

\usepackage{amsmath, amssymb, amsthm}
\DeclareMathOperator{\EX}{\mathbb{E}}
\DeclareMathOperator{\Prob}{\mathbb{P}}
\DeclareMathOperator*{\argmax}{arg\,max}

\usepackage{xcolor}

\newtheorem*{assumption}{Assumption}
\newtheorem*{lemma}{Lemma}
%%% END Article customizations

%%% The "real" document content comes below...

\title{Advertising and Pricing}
\author{Gabriele Daglio, Federico Di Cesare, Jacopo Germano}
%\date{} % Activate to display a given date or no date (if empty),
         % otherwise the current date is printed 

\begin{document}
\maketitle

\section{The setting}

Advertising is used to attract users on an ecommerce website that sells only one type of item. Each user belongs to one of three classes, and the classes are determined by the combination of two binary features. For each class $c \in C$ the following functions are modeled:
\begin{itemize}
\item A stochastic number of daily clicks of new users (i.e., that have never clicked before on the ads), represented by the discrete random variable $N_{c,b}\sim NewClicks(c,b)$, such that $\EX[N_{c,b}]=n(c,b)$, where $b \in B$ is a bid value. 
\item A conversion rate function providing the probability that a user will buy the item at a certain price,  $r(c,p)$, where $p \in P$ is the price. For each user of class $c$ that has clicked on the ad, a Bernoulli random variable $D_{c,p} \sim Bern(r(c,p))$ indicates whether the user bought the product ($D_{c,p}=1$) or not ($D_{c,p}=0$), such that $\EX[D_{c,p}]=r(c,p)$. In the code, such distribution is called $ClickConverted(c,p)$.
\item A probability distribution $FutureVisits(c)$ over the number of times a user of class $c$ will come back to the ecommerce website to buy that item by 30 days after the first purchase. In other words, when a user makes a purchase, they are somehow likely to make more purchases in the near future, and after that they will leave the website forever. For each user of class $c$, a discrete random variable $F_c \sim FutureVisits(c)$ indicates the number of times that the user came back, and for each class $c$ the function $f(c)$ is defined such that $\EX[F_c] = f(c)$
\item A probability distribution $CostPerClick(c,b)$. For each click, the random variable $C_{c,b}\sim CostPerClick(c,b)$ represents the amount that is paid to the ad publisher, such that $\EX[C_{c,b}] = k(c,b)$ and $\Prob(C_{c,b} \leq b) = 1$.
\end{itemize}

A margin function $m(p)$, where $p \in P$ is a price, is available to indicate how much profit is obtained if the an item is sold at the price $p$. 


\subsection{New Daily Clicks $NewClicks(c,b)$}
The $NewClicks(c,b)$ distribution is used to sample the number of daily clicks from users of class $c$ if the bid $b$ is used.
\newline
\newline
We adopt the following assumption about the distribution: 
\begin{assumption}[Agnostic Publisher] The mean of $NewClicks(c,b)$, $n(c,b)$, can be expressed as:
\begin{align*}
n(c,b)=n(c)v(b)
\end{align*}
Where $n(c)$ does not depend on $b$ and $v(b)$ does not depend on $c$.
\end{assumption}
In the assumption above, $v(b)$ represents the fact that higher bids will win more auctions and $n(c)$ represents the fact that, for each class, a different percentage of the users will belong to class $c$ \textit{and} click on the ad. The meaning of the Agnostic Publisher assumption is that a change in the bid will change the number of users seeing the ad but will never change the percentage of users for each class. In other words, an increase in the auctions won will reflect in an increase in the number of clicks with the same proportion on all the classes. The reason behind the name is that in every auction the average competition does not depend on the user class, because the Publisher does not know anything about the class of the user (assuming that the Publisher is choosing every aspect of when and where to display the ad).
\newline
\newline
In our implementation we defined:
\begin{itemize}
\item $n(c)=n_c$, for each class a constant $n_c > 0$ is randomly sampled when the environment is generated.
\item $v(b)=\frac{1}{1+e^{-\overline z(b-\overline b)}}$, where $\overline z > 0$ and $\overline b > 0$ are randomly sampled when the environment is generated. The function $v(b)$ is the same for all the classes.
\end{itemize}
And we defined the distribution as
{\color{red}Come la abbiamo definita}
\subsection{Conversion Rate $r(c,p)$}
For each class $c \in C$, the function $r_c(p)$ used to model the conversion rate of users belonging to that class must have the following properties:
\begin{itemize}
\item $r_c(0) \approx 1$: The user will be very likely to buy the product if it comes for free.
\item $\lim\limits_{p \to +\infty} r_c(p) = 0$: As the price goes to infinity, the probability that the user will buy it goes to zero.
\item $r_c(p)$ is monotonically decreasing with respect to p: an increase of the price will never increase the probability that the user will buy it.
\end{itemize}
The function $r(c,p)$ is then defined as: $r(c,p) := r_c(p)$.
Despite the fact that the functions $r_c(p)$ could be in principle defined in completely different ways, in our implementation we chose to use only (reflected, translated and horizontally scaled) sigmoidal functions:
\begin{align*}
r_c(p) = \frac{1}{1+e^{-z_c(P_c-p)}}
\end{align*}
Where $P_c$, the inflection point of the sigmoid, can be seen as the average reserve price of the users of the class and $z_c$ can be seen as the concentration of the reserve prices of the users around the average: if $z_c$ is small the reserve prices of the many users will be more distributed across the domain and the function will be more flat, while as $z_c$ grows the reserve prices of the many users will be more concentrated around the average and at the point $P_c$ there will be a rapid transition from "buy" to "don't buy".
\subsection{Future Visits $FutureVisits(c)$}
{\color{red}Descrizione di come la abbiamo implementata.}

\subsection{Cost per click $CostPerClick(c,b)$}
The $CostPerClick$ distribution is used to sample the cost that is paid to the ad publisher whenever a user clicks on the ad. Because of the  auction mechanism and the unknown bids and budget constraints of the competitors, it could be smaller than the bid by a stochastic amount.
We adopt the following assumption: 
\begin{assumption}[Class-independent cost per click] The probability distribution $CostPerClick(c,b)$ does not depend on c, that is:
\begin{align*}
CostPerClick(c,b)=CostPerClick(b)\\
k(c,b) = k(b)
\end{align*}
\end{assumption}
In our implementation we chose to keep it constant and equal to the bid, that is:
\begin{align*}
\Prob(C_{c,b} = k(c,b)) = 1\\
k(c,b) = b
\end{align*}
\subsection{Margin function $m(p)$}
{\color{red}Come la abbiamo definita}
\section{Step 1}
The goal is to maximize the expected profit over a single day, where the future visits of a user are considered to contribute in expected value to the profit of the day of the first visit.
\newline
\newline
For each class $c$, we define:
\begin{itemize}
\item The random variable $N_{c,b}$ representing the number of new clicks of users of class $c$, such that $N_{c,b} \sim NewClicks(c,b)$
\item The sequence $(C_{b,i})_{i=1,...,N_{c,b}}$ of random variables representing the cost paid for each click $i$, such that $C_{b,i}\sim CostPerClick(b)$
\item The sequence $(D_{c,p,i})_{i=1,...,N_{c,b}}$ of random variables representing whether user $i$ of class $c$ purchased the item, such that $D_{c,p,i}\sim Bern(r(c,p))$
\item The sequence $(F_{c,i})_{i=1,...,N_{c,b}}$ of random variables representing the number of future visits of the user $i$ of class $c$, such that $F_{c,i}\sim FutureVisits(c)$
\end{itemize}
We can express the expected profit as follows:
\begin{align*}
ExpectedProfit(p,b) = \EX\left[\sum_{c \in C}{\sum_{i =1}^{N_{c,b}}{\bigg( D_{c,p,i}(1+F_{c,i})m(p)-C_{b,i}\bigg)}}\right]
\end{align*}
And therefore formulate the optimization problem as follows:
\begin{align*}
\argmax_{p,b}{ExpectedProfit(p,b)}
\end{align*}

With some manipulations using the properties of the expected value (Appendix \ref{sec.DerExpProf}), the expected profit can be expressed as follows:

\begin{align*}
ExpectedProfit(p,b)=m(p)\sum_{c \in C}{n(c,b)r(c,p)(1+f(c))}-k(b)\sum_{c \in C}{n(c,b)}
\end{align*}

Obtaining the following formulation, that depends only on the means of the distributions:

\begin{align*}
\argmax_{p,b}{m(p)\sum_{c \in C}{n(c,b)r(c,p)(1+f(c))}-k(b)\sum_{c \in C}{n(c,b)}}
\end{align*}

Under the Agnostic Publisher assumption, the following interesting result holds:

\begin{lemma}[Bid Independent Price Hierarchy]
If $ExpectedProfit(p_1,\overline b) \ge ExpectedProfit(p_2,\overline b)$ for some bid value $\overline b$,  then $ExpectedProfit(p_1,\overline b) \ge ExpectedProfit(p_2,\overline b)$  for every possible bid value $b'$.
\end{lemma}
\begin{proof}
see Appendix \ref{sec.BIPHLProof}.
\end{proof}

As a consequence of this result, we developed an algorithm that:
\begin{itemize}
\item Finds the optimal price $p^*\in P$ that maximizes $ExpectedProfit(p,\overline b)$ for a fixed bid value $\overline b$ (We take the median of the set of possible values). The time complexity of this step is $O(|P|)$
\item Finds the optimal bid value $b^* \in B$ that maximizes $ExpectedProfit(p^*, b)$, using the optimal price $p^*$ found at step 1. The time complexity of this step is $O(|B|)$
\item The solution is $(p^*, b^*)$.
\end{itemize}

\section{Step 2}
We model the online learning problem as follows:
\begin{itemize}
\item The time steps are the single days. We consider the days as numbered from 1 to the learning horizon $H$.
\item The learning horizon $H$ is 365.
\item The learner will try a sequence of prices $(p_1, ..., p_H)$, $p_i \in P$.
\item The learner will try a sequence of bids $(b_1, ..., b_H)$, $b_i \in B$.
\item The full feedback of time step $t$ (the total profit generated from the users that clicked on the ad for the first time on time step $t$) is received after a delay of 30 time steps. The full feedback of time step $t$ is the sum of different profits that can come with different delays with respect to $t$.
\item Until time step 30, the learner will use all the partial feedbacks of the previous time steps as if they were complete. After that, the feedbacks will always be complete.
\end{itemize}

The objective function to maximize is the cumulative expected profit over the learning horizon $H$:

\begin{align*}
\EX\left[\sum_{t=1}^H{\sum_{c \in C}{n(c,b_t)r(c,p_t)m(p_t)(1+\EX[f(c)])-n(c,b_t)cost}}\right]
\end{align*}

\appendix
\section{Derivation of ExpectedProfit}\label{sec.DerExpProf}
\begin{align*}
ExpectedProfit(p,b) 
&= \EX\left[\sum_{c \in C}{\sum_{i =1}^{N_{c,b}}{\bigg( D_{c,p,i}(1+F_{c,i})m(p)-C_{b,i}\bigg)}}\right]\\
&= \EX\left[\sum_{c \in C}{\bigg(\sum_{i =1}^{N_{c,b}}{ D_{c,p,i}(1+F_{c,i})m(p)}-\sum_{i =1}^{N_{c,b}}C_{b,i}\bigg)}\right]\\
&= \EX\left[\sum_{c \in C}{\sum_{i =1}^{N_{c,b}}{ D_{c,p,i}(1+F_{c,i})m(p)}}\right]-\sum_{c \in C}{\EX\left[\sum_{i =1}^{N_{c,b}}C_{b,i}\right]}\\
&= \EX\left[\sum_{c \in C}{\sum_{i =1}^{N_{c,b}}{ D_{c,p,i}(1+F_{c,i})m(p)}}\right]-\sum_{c \in C}{\EX\left[\EX\left[\sum_{i =1}^{N_{c,b}}C_{b,i}\bigg|N_{c,b}\right]\right]}\\
&= \EX\left[\sum_{c \in C}{\sum_{i =1}^{N_{c,b}}{ D_{c,p,i}(1+F_{c,i})m(p)}}\right]-\sum_{c \in C}{\EX\left[\sum_{i =1}^{N_{c,b}}\EX\left[C_{b,i}\bigg|N_{c,b}\right]\right]}\\
&= \EX\left[\sum_{c \in C}{\sum_{i =1}^{N_{c,b}}{ D_{c,p,i}(1+F_{c,i})m(p)}}\right]-\sum_{c \in C}{\EX\left[\sum_{i =1}^{N_{c,b}}k(c,b)\right]}\\
&= \EX\left[\sum_{c \in C}{\sum_{i =1}^{N_{c,b}}{ D_{c,p,i}(1+F_{c,i})m(p)}}\right]-\sum_{c \in C}{\EX\left[N_{c,b}k(c,b)\right]}\\
&= \EX\left[\sum_{c \in C}{\sum_{i =1}^{N_{c,b}}{ D_{c,p,i}(1+F_{c,i})m(p)}}\right]-\sum_{c \in C}{\EX[N_{c,b}]k(b)}\\
&= m(p)\sum_{c \in C}{\EX\left[\sum_{i =1}^{N_{c,b}}{ D_{c,p,i}(1+F_{c,i})}\right]}-\sum_{c \in C}{n(c,b)k(b)}\\
&= m(p)\sum_{c \in C}{\EX\left[\EX\left[\sum_{i =1}^{N_{c,b}}{ D_{c,p,i}(1+F_{c,i})}\Bigg|N_{c,b}\right]\right]}-k(b)\sum_{c \in C}{n(c,b)}\\
&= m(p)\sum_{c \in C}{\EX\left[\sum_{i =1}^{N_{c,b}}{\EX\left[D_{c,p,i}(1+F_{c,i})\bigg|N_{c,b}\right]}\right]}-k(b)\sum_{c \in C}{n(c,b)}\\
&= m(p)\sum_{c \in C}{\EX\left[\sum_{i =1}^{N_{c,b}}{\EX\left[D_{c,p,i}\bigg|N_{c,b}\right]\EX\left[(1+F_{c,i})\bigg|N_{c,b}\right]}\right]}-k(b)\sum_{c \in C}{n(c,b)}\\
&= m(p)\sum_{c \in C}{\EX\left[\sum_{i =1}^{N_{c,b}}{\EX\left[D_{c,p,i}\right]\EX\left[(1+F_{c,i})\right]}\right]}-k(b)\sum_{c \in C}{n(c,b)}\\
&= m(p)\sum_{c \in C}{\EX\left[\sum_{i =1}^{N_{c,b}}{r(c,p)\Big(1+f(c)\Big)}\right]}-k(b)\sum_{c \in C}{n(c,b)}\\
&= m(p)\sum_{c \in C}{\EX\left[N_{c,b}r(c,p)\Big(1+f(c)\Big)\right]}-k(b)\sum_{c \in C}{n(c,b)}\\
&= m(p)\sum_{c \in C}{\EX\left[N_{c,b}\right]r(c,p)\Big(1+f(c)\Big)}-k(b)\sum_{c \in C}{n(c,b)}\\
&= m(p)\sum_{c \in C}{n(c,b)r(c,p)\Big(1+f(c)\Big)}-k(b)\sum_{c \in C}{n(c,b)}
\end{align*}

\section{Proof of the Bid Independent Price Hierarchy lemma}\label{sec.BIPHLProof}
\begin{align*}
ExpectedProfit(p,b)=\sum_{c \in C}{n(c,b)r(c,p)m(p)(1+f(c))}-k(b)\sum_{c \in C}{n(c,b)}
\end{align*}
\begin{align*}
\sum_{c \in C}{n(c,b)r(c,p_1)m(p_1)(1+f(c))}-k(b)\sum_{c \in C}{n(c,b)}  &\ge \sum_{c \in C}{n(c,b)r(c,p_2)m(p_2)(1+f(c))}-k(b)\sum_{c \in C}{n(c,b)}\\
\sum_{c \in C}{n(c,b)r(c,p_1)m(p_1)(1+f(c))}  &\ge \sum_{c \in C}{n(c,b)r(c,p_2)m(p_2)(1+f(c))}\\
\sum_{c \in C}{n(c)v(b)r(c,p_1)m(p_1)(1+f(c))}  &\ge \sum_{c \in C}{n(c)v(b)r(c,p_2)m(p_2)(1+f(c))}\\
v(b)m(p_1)\sum_{c \in C}{n(c)r(c,p_1)(1+f(c))}  &\ge v(b)m(p_2)\sum_{c \in C}{n(c)r(c,p_2)(1+f(c))}\\
m(p_1)\sum_{c \in C}{n(c)r(c,p_1)(1+f(c))}  &\ge m(p_2)\sum_{c \in C}{n(c)r(c,p_2)(1+f(c))}
\end{align*}
Therefore the relation holds for every possible value of the bid.
\end{document}
